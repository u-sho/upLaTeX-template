\documentclass[uplatex,dvipdfmx,11pt]{jsbook}
\usepackage[%margin=25.04truemm
 textwidth=45zw,
 lines=40,
 centering
]{geometry}
\usepackage{u-sho_style}
\usepackage{u-sho_math}

\title{
  {\normalsize up\LaTeXe テンプレート}% chktex 1
  \\[10truemm]
  jsbookクラスを用いた\\
  ushoStyleパッケージの見本
  \vspace{30truemm}
}
\author{
  \large{電気通信大学 情報理工学域}\\
  \large{Ⅱ類 電子情報学プログラム}
  \\[10truemm]
  u-sho(上地 将平)
  \\[20truemm]
  参考資料
  \\[5truemm]
  特になし
  \vspace{20truemm}
}
\date{作成日: \和暦\today}

\begin{document}
  \pagenumbering{roman}
  \maketitle
  \tableofcontents

  \chapter{目的}%
  \pagenumbering{arabic}
    知らんが

  \chapter{概要}%
    参考文献~\cite{キー1}

  \chapter{鳴き声} %chktex 19

    \section{にゃあにゃあ}
      なんか

      \subsection{\colorbox{orange}{\textcolor{white}{がお}}がお}
        すると

    \section{ぴえん} % chktex 19

      \subsection{ぱおーん}

        \fcolorbox{pink}[HTML]{555555}{\textcolor{white}{ぱおーん}}
        \tbref{tb:result}や\fgref{fg:paopao}を \verb|\tbref| と \verb|\fgref| コマンドで引いてみた

  \newpage

  \chapter{結果および考察}

    \section{がおがお}

      しらんけど\num{6e2}回サイコロふった.初期位相\ang{-66;5;4}だったとかなんとか

      \begin{table}[h]
        \centering
          \caption{サイコロを振った結果(\(N=600\))}\label{tb:result}
          \begin{tabular}{%
            c|S[table-format=3.0]cS[table-format=+1.1e+1,table-figures-uncertainty=1,color=orange]%
          }
            \hline
            出た目 & 回数\(n\) & 割合\(n/N\) & {\color{orange}\((n/N - 1/6)\)} \\
            \hline
            1      &  88       & 0.147       & -2.0+-1  e-2 \\
            2      & 102       & 0.170       &  3  +-0.1e-3 \\
            \cline{2-4}
            3      &  84       & 0.140       & -2.7     e-2 \\
            4      & 114       & 0.190       &  2.3     e-2 \\
            5      & 109       & 0.182       &  1.5     e-2 \\
            6      & 103       & 0.172       & \color{red}5e-3 \\
            \hline
          \end{tabular}
      \end{table}

    \section{ぱおぱお}

      これはなんの\emph{ず}?pyo
        \begin{figure}[!ht]
          \centering
            \includegraphics{img/350x150.png} %chktex 29
            \caption{ろんりかいろ}\label{fg:paopao}
        \end{figure}
      \numlist{1;2;10}番目のわちゃわちゃ\\
      \(1,\,2,\,\ldots\,,\,10\)あるいは\(1,\,2\dotsc 10\).\(\llistto{Y}{m}\)は \(\llistto{X}{n}\)
      \[
        t
      \]

  \newpage

      あいうえおかきくけこさしすせそたちつてとなにぬねのはひふへほまみむめもやいゆえよらりるれ1あいうえおかきくけこさしすせそたちつてとなにぬねのはひふへほまみむめもやいゆえよらりるれ2あいうえおかきくけこさしすせそたちつてとなにぬねのはひふへほまみむめもやいゆえよらりるれ3あいうえおかきくけこさしすせそたちつてとなにぬねのはひふへほまみむめもやいゆえよらりるれ4あいうえおかきくけこさしすせそたちつてとなにぬねのはひふへほまみむめもやいゆえよらりるれ5あいうえおかきくけこさしすせそたちつてとなにぬねのはひふへほまみむめもやいゆえよらりるれ6あいうえおかきくけこさしすせそたちつてとなにぬねのはひふへほまみむめもやいゆえよらりるれ7あいうえおかきくけこさしすせそたちつてとなにぬねのはひふへほまみむめもやいゆえよらりるれ8あいうえおかきくけこさしすせそたちつてとなにぬねのはひふへほまみむめもやいゆえよらりるれ9あいうえおかきくけこさしすせそたちつてとなにぬねのはひふへほまみむめもやいゆえよらりるれ十あいうえおかきくけこさしすせそたちつてとなにぬねのはひふへほまみむめもやいゆえよらりるれ1あいうえおかきくけこさしすせそたちつてとなにぬねのはひふへほまみむめもやいゆえよらりるれ2あいうえおかきくけこさしすせそたちつてとなにぬねのはひふへほまみむめもやいゆえよらりるれ3あいうえおかきくけこさしすせそたちつてとなにぬねのはひふへほまみむめもやいゆえよらりるれ4あいうえおかきくけこさしすせそたちつてとなにぬねのはひふへほまみむめもやいゆえよらりるれ5あいうえおかきくけこさしすせそたちつてとなにぬねのはひふへほまみむめもやいゆえよらりるれ6あいうえおかきくけこさしすせそたちつてとなにぬねのはひふへほまみむめもやいゆえよらりるれ7あいうえおかきくけこさしすせそたちつてとなにぬねのはひふへほまみむめもやいゆえよらりるれ8あいうえおかきくけこさしすせそたちつてとなにぬねのはひふへほまみむめもやいゆえよらりるれ9あいうえおかきくけこさしすせそたちつてとなにぬねのはひふへほまみむめもやいゆえよらりるれ二十あいうえおかきくけこさしすせそたちつてとなにぬねのはひふへほまみむめもやいゆえよらりる1あいうえおかきくけこさしすせそたちつてとなにぬねのはひふへほまみむめもやいゆえよらりるれ2あいうえおかきくけこさしすせそたちつてとなにぬねのはひふへほまみむめもやいゆえよらりるれ3あいうえおかきくけこさしすせそたちつてとなにぬねのはひふへほまみむめもやいゆえよらりるれ4あいうえおかきくけこさしすせそたちつてとなにぬねのはひふへほまみむめもやいゆえよらりるれ5あいうえおかきくけこさしすせそたちつてとなにぬねのはひふへほまみむめもやいゆえよらりるれ6あいうえおかきくけこさしすせそたちつてとなにぬねのはひふへほまみむめもやいゆえよらりるれ7あいうえおかきくけこさしすせそたちつてとなにぬねのはひふへほまみむめもやいゆえよらりるれ8あいうえおかきくけこさしすせそたちつてとなにぬねのはひふへほまみむめもやいゆえよらりるれ9あいうえおかきくけこさしすせそたちつてとなにぬねのはひふへほまみむめもやいゆえよらりるれ三十あいうえおかきくけこさしすせそたちつてとなにぬねのはひふへほまみむめもやいゆえよらりる1あいうえおかきくけこさしすせそたちつてとなにぬねのはひふへほまみむめもやいゆえよらりるれ2あいうえおかきくけこさしすせそたちつてとなにぬねのはひふへほまみむめもやいゆえよらりるれ3あいうえおかきくけこさしすせそたちつてとなにぬねのはひふへほまみむめもやいゆえよらりるれ4あいうえおかきくけこさしすせそたちつてとなにぬねのはひふへほまみむめもやいゆえよらりるれ5あいうえおかきくけこさしすせそたちつてとなにぬねのはひふへほまみむめもやいゆえよらりるれ6あいうえおかきくけこさしすせそたちつてとなにぬねのはひふへほまみむめもやいゆえよらりるれ7あいうえおかきくけこさしすせそたちつてとなにぬねのはひふへほまみむめもやいゆえよらりるれ8あいうえおかきくけこさしすせそたちつてとなにぬねのはひふへほまみむめもやいゆえよらりるれ9あいうえおかきくけこさしすせそたちつてとなにぬねのはひふへほまみむめもやいゆえよらりるれ四十あいうえおかきくけこさしすせそたちつてとなにぬねのはひふへほまみむめもやいゆえよらりる1あいうえおかきくけこさしすせそたちつてとなにぬねのはひふへほまみむめもやいゆえよらりるれ2あいうえおかきくけこさしすせそたちつてとなにぬねのはひふへほまみむめもやいゆえよらりるれ3あいうえおかきくけこさしすせそたちつてとなにぬねのはひふへほまみむめもやいゆえよらりるれ4あいうえおかきくけこさしすせそたちつてとなにぬねのはひふへほまみむめもやいゆえよらりるれ5あいうえおかきくけこさしすせそたちつてとなにぬねのはひふへほまみむめもやいゆえよらりるれ6あいうえおかきくけこさしすせそたちつてとなにぬねのはひふへほまみむめもやいゆえよらりるれ7あいうえおかきくけこさしすせそたちつてとなにぬねのはひふへほまみむめもやいゆえよらりるれ8あいうえおかきくけこさしすせそたちつてとなにぬねのはひふへほまみむめもやいゆえよらりるれ9あいうえおかきくけこさしすせそたちつてとなにぬねのはひふへほまみむめもやいゆえよらりるれ五十あいうえおかきくけこさしすせそたちつてとなにぬねのはひふへほまみむめもやいゆえよらりる

      \subsection{にゅ〜ぺーじ}

        ひるべるとくうかん \(\lrp[\big]{\H,\L{\H}} \lrb[\Big]{x,y}\):
        \[
          \lrs*{\frac{3}{100}}\times \mathcal{X}\inp*{\frac{100}{3}}{3+i} %\hsp*{3+i}{\frac{4}{101}}
          \Exist{ρ},\All{\rho}, \vphantom{\rho}^\forall\!\rho \bigl\langle\!\!\bigl\langle
        \]
        あるいは

  \clearpage

  \begin{thebibliography}{9}
    \bibitem{キー1} 参考文献の名前・著者1 available at \url{https://github.com/u-sho/upLaTeX-template}
  \end{thebibliography}

\end{document}

