\documentclass[uplatex,dvipdfmx,11pt]{jsbook}
\usepackage[%margin=25.04truemm
 textwidth=45zw,
 lines=40,
 centering
]{geometry}
\usepackage{u-sho_style}
\usepackage{u-sho_math}
\hypersetup{
  pdfinfo={
    CreationDate={D:20210101124246+09'00'},
    ModDate={\pdffilemoddate{\jobname}},
    Subject={upLaTeX template with u-sho's packages},
    Keywords={upLaTeX template;jsbook;u-sho math;u-sho style}
  }
}

\title{
  {\normalsize up\LaTeXe テンプレート}% chktex 1
  \\[10truemm]
  jsbookクラスを用いた\\
  ushoStyleパッケージの見本
  \texorpdfstring{\vspace*{30truemm}}{}
}
\author{
  \large{電気通信大学 大学院情報理工学研究科}\\
  \large{情報・ネットワーク工学専攻}
  \\[10truemm]
  u-sho(上地 将平)
  \\[20truemm]
  参考資料
  \\[5truemm]
  特になし
  \texorpdfstring{\vspace*{20truemm}}{}
}
\date{作成日:令和3年1月1日\\最終更新日:\和暦\today}%chktex 19

\begin{document}
  \maketitle

  \pagenumbering{roman}\tableofcontents

  \chapter{目的}%
  \pagenumbering{arabic}
    知らんが

  \chapter{概要}%
    参考文献~\cite{キー1}

  \chapter{鳴き声} %chktex 19

    \section{にゃあにゃあ}
      なんか

      \subsection{\texorpdfstring{\bgOrange{\tWhite{がお}}がお}{がおがお}}
        すると

    \section{ぴえん} % chktex 19

      \subsection{ぱおーん}

        \fcolorbox{pink}[HTML]{555555}{\tWhite{ぱおーん}}
        \prettyref{th:center}や\prettyref{tb:result},\prettyref{fg:paopao},\prettyref{eq:newpage}を \verb|\prettyref| コマンドで引いてみた

  \newpage

  \chapter{結果および考察}

    \section{がおがお}

      しらんけど\num{6e2}回サイコロふった.初期位相\ang{-66;5;4}だったとかなんとか

      \begin{ctable}[h]
        \caption{サイコロを振った結果(\(N=600\))}\label{tb:result}
        \begin{tabular}{%
          c|S[table-format=3.0]cS[table-format=+1.1e+1,table-figures-uncertainty=1,color=orange]%
        }
          \hline
          出た目 & 回数\(n\) & 割合\(n/N\) & \(\tOrange{(n/N - 1/6)}\) \\
          \hline
          1      &  88       & 0.147       & -2.0+-1  e-2 \\
          2      & 102       & 0.170       &  3  +-0.1e-3 \\
          \cline{2-4}
          3      &  84       & 0.140       & -2.7     e-2 \\
          4      & 114       & 0.190       &  2.3     e-2 \\
          5      & 109       & 0.182       &  1.5     e-2 \\
          6      & 103       & 0.172       & \color{red}5e-3 \\
          \hline
        \end{tabular}
      \end{ctable}

    \section{ぱおぱお}

      これはなんの\emph{ず}?pyo
        \begin{figure}[!ht]
          \centering
            \includegraphics{img/350x150.png} %chktex 29
            \caption{ろんりかいろ}\label{fg:paopao}
        \end{figure}
      \numlist{1;2;10}番目のわちゃわちゃ\\
      \(1,\,2,\,\ldots\,,\,10\)あるいは\(1,\,2\dotsc 10\).\(\llistto{Y}{m}\)は \(\llistto{X}{n}\)
      \[
        t
      \]

  \newpage

      あいうえおかきくけこさしすせそたちつてとなにぬねのはひふへほまみむめもやいゆえよらりるれ1あいうえおかきくけこさしすせそたちつてとなにぬねのはひふへほまみむめもやいゆえよらりるれ2あいうえおかきくけこさしすせそたちつてとなにぬねのはひふへほまみむめもやいゆえよらりるれ3あいうえおかきくけこさしすせそたちつてとなにぬねのはひふへほまみむめもやいゆえよらりるれ4あいうえおかきくけこさしすせそたちつてとなにぬねのはひふへほまみむめもやいゆえよらりるれ5あいうえおかきくけこさしすせそたちつてとなにぬねのはひふへほまみむめもやいゆえよらりるれ6あいうえおかきくけこさしすせそたちつてとなにぬねのはひふへほまみむめもやいゆえよらりるれ7あいうえおかきくけこさしすせそたちつてとなにぬねのはひふへほまみむめもやいゆえよらりるれ8あいうえおかきくけこさしすせそたちつてとなにぬねのはひふへほまみむめもやいゆえよらりるれ9あいうえおかきくけこさしすせそたちつてとなにぬねのはひふへほまみむめもやいゆえよらりるれ十あいうえおかきくけこさしすせそたちつてとなにぬねのはひふへほまみむめもやいゆえよらりるれ1あいうえおかきくけこさしすせそたちつてとなにぬねのはひふへほまみむめもやいゆえよらりるれ2あいうえおかきくけこさしすせそたちつてとなにぬねのはひふへほまみむめもやいゆえよらりるれ3あいうえおかきくけこさしすせそたちつてとなにぬねのはひふへほまみむめもやいゆえよらりるれ4あいうえおかきくけこさしすせそたちつてとなにぬねのはひふへほまみむめもやいゆえよらりるれ5あいうえおかきくけこさしすせそたちつてとなにぬねのはひふへほまみむめもやいゆえよらりるれ6あいうえおかきくけこさしすせそたちつてとなにぬねのはひふへほまみむめもやいゆえよらりるれ7あいうえおかきくけこさしすせそたちつてとなにぬねのはひふへほまみむめもやいゆえよらりるれ8あいうえおかきくけこさしすせそたちつてとなにぬねのはひふへほまみむめもやいゆえよらりるれ9あいうえおかきくけこさしすせそたちつてとなにぬねのはひふへほまみむめもやいゆえよらりるれ二十あいうえおかきくけこさしすせそたちつてとなにぬねのはひふへほまみむめもやいゆえよらりる1あいうえおかきくけこさしすせそたちつてとなにぬねのはひふへほまみむめもやいゆえよらりるれ2あいうえおかきくけこさしすせそたちつてとなにぬねのはひふへほまみむめもやいゆえよらりるれ3あいうえおかきくけこさしすせそたちつてとなにぬねのはひふへほまみむめもやいゆえよらりるれ4あいうえおかきくけこさしすせそたちつてとなにぬねのはひふへほまみむめもやいゆえよらりるれ5あいうえおかきくけこさしすせそたちつてとなにぬねのはひふへほまみむめもやいゆえよらりるれ6あいうえおかきくけこさしすせそたちつてとなにぬねのはひふへほまみむめもやいゆえよらりるれ7あいうえおかきくけこさしすせそたちつてとなにぬねのはひふへほまみむめもやいゆえよらりるれ8あいうえおかきくけこさしすせそたちつてとなにぬねのはひふへほまみむめもやいゆえよらりるれ9あいうえおかきくけこさしすせそたちつてとなにぬねのはひふへほまみむめもやいゆえよらりるれ三十あいうえおかきくけこさしすせそたちつてとなにぬねのはひふへほまみむめもやいゆえよらりる1あいうえおかきくけこさしすせそたちつてとなにぬねのはひふへほまみむめもやいゆえよらりるれ2あいうえおかきくけこさしすせそたちつてとなにぬねのはひふへほまみむめもやいゆえよらりるれ3あいうえおかきくけこさしすせそたちつてとなにぬねのはひふへほまみむめもやいゆえよらりるれ4あいうえおかきくけこさしすせそたちつてとなにぬねのはひふへほまみむめもやいゆえよらりるれ5あいうえおかきくけこさしすせそたちつてとなにぬねのはひふへほまみむめもやいゆえよらりるれ6あいうえおかきくけこさしすせそたちつてとなにぬねのはひふへほまみむめもやいゆえよらりるれ7あいうえおかきくけこさしすせそたちつてとなにぬねのはひふへほまみむめもやいゆえよらりるれ8あいうえおかきくけこさしすせそたちつてとなにぬねのはひふへほまみむめもやいゆえよらりるれ9あいうえおかきくけこさしすせそたちつてとなにぬねのはひふへほまみむめもやいゆえよらりるれ四十あいうえおかきくけこさしすせそたちつてとなにぬねのはひふへほまみむめもやいゆえよらりる1あいうえおかきくけこさしすせそたちつてとなにぬねのはひふへほまみむめもやいゆえよらりるれ2あいうえおかきくけこさしすせそたちつてとなにぬねのはひふへほまみむめもやいゆえよらりるれ3あいうえおかきくけこさしすせそたちつてとなにぬねのはひふへほまみむめもやいゆえよらりるれ4あいうえおかきくけこさしすせそたちつてとなにぬねのはひふへほまみむめもやいゆえよらりるれ5あいうえおかきくけこさしすせそたちつてとなにぬねのはひふへほまみむめもやいゆえよらりるれ6あいうえおかきくけこさしすせそたちつてとなにぬねのはひふへほまみむめもやいゆえよらりるれ7あいうえおかきくけこさしすせそたちつてとなにぬねのはひふへほまみむめもやいゆえよらりるれ8あいうえおかきくけこさしすせそたちつてとなにぬねのはひふへほまみむめもやいゆえよらりるれ9あいうえおかきくけこさしすせそたちつてとなにぬねのはひふへほまみむめもやいゆえよらりるれ五十あいうえおかきくけこさしすせそたちつてとなにぬねのはひふへほまみむめもやいゆえよらりる

      \subsection{にゅ〜ぺーじ}

        ひるべるとくうかん \(\lrp[\big]{\H,\L\H} \lrb[\Big]{x,y}\):
        \begin{equation}\label{eq:newpage}
          \sfrac{3}{100}\times \mathcal{X}\inp*{\wfrac{100}{3}}{3+i} %\hsp*{3+i}{\frac{4}{101}}
          \Exist{ρ},\All{\rho}, \vphantom{\rho}^\forall\!\rho \bigl\langle\!\!\bigl\langle
        \end{equation}
        あるいは
        \begin{theorem}[中間値の定理]\label{th:center}
          閉区間\(\closedInterval{a,b}\subset\R\)について連続な関数\(f:\R\to\R\)を考える.
          \(f\lrp{a}\leq f\lrp{b}\)であるとき,
          \(\openInterval{f\lrp{a},f\lrp{b}}\)上の任意の\(k\in\R\)に対して,
          \(f\lrp{c}=k\)なる\(c\in\openInterval{a,b}\)が存在し,
          \(f\lrp{a}>f\lrp{b}\)であるとき,
          \(\openInterval{f\lrp{b},f\lrp{a}}\)上の任意の\(k\in\R\)に対して,
          \(f\lrp{c}=k\)なる\(c\in\openInterval{b,a}\)が存在する.
        \end{theorem}
        \begin{proof}
          明らか.
        \end{proof}

  \clearpage

  \begin{thebibliography}{9}
    \bibitem{キー1} 参考文献の名前・著者1 available at \url{https://github.com/u-sho/upLaTeX-template}
  \end{thebibliography}

\end{document}
